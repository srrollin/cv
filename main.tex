%%%%%%%%%%%%%%%%%%%%%%%%%%%%%%%%%%%%%%%%%
% "ModernCV" CV and Cover Letter
% LaTeX Template
% Version 1.1 (9/12/12)
%
% This template has been downloaded from:
% http://www.LaTeXTemplates.com
%
% Original author:
% Xavier Danaux (xdanaux@gmail.com)
%
% License:
% CC BY-NC-SA 3.0 (http://creativecommons.org/licenses/by-nc-sa/3.0/)
%
% Important note:
% This template requires the moderncv.cls and .sty files to be in the same 
% directory as this .tex file. These files provide the resume style and themes 
% used for structuring the document.
%
%%%%%%%%%%%%%%%%%%%%%%%%%%%%%%%%%%%%%%%%%

%----------------------------------------------------------------------------------------
%	PACKAGES AND OTHER DOCUMENT CONFIGURATIONS
%----------------------------------------------------------------------------------------

\documentclass[10pt,a4paper,sans]{moderncv} % Font sizes: 10, 11, or 12; paper sizes: a4paper, letterpaper, a5paper, legalpaper, executivepaper or landscape; font families: sans or roman

\moderncvstyle{classic} % CV theme - options include: 'casual' (default), 'classic', 'oldstyle' and 'banking'
\moderncvcolor{green} % CV color - options include: 'blue' (default), 'orange', 'green', 'red', 'purple', 'grey' and 'black'

\usepackage{lipsum} % Used for inserting dummy 'Lorem ipsum' text into the template

\usepackage[scale=0.75]{geometry} % Reduce document margins

%----------------------------------------------------------------------------------------
%	NAME AND CONTACT INFORMATION SECTION
%----------------------------------------------------------------------------------------

\firstname{Fellipe} % Your first name
\familyname{Rollin} % Your last name

% All information in this block is optional, comment out any lines you don't need
\title{Sr Software Engineer}
\address{Stockholm - Sweden}{\href{https://www.linkedin.com/in/felliperollin}{linkedin.com/in/felliperollin}}
\mobile{+46 (76) 3185722}
\email{fellipetr@gmail.com}

%----------------------------------------------------------------------------------------

\begin{document}

\makecvtitle % Print the CV title

%----------------------------------------------------------------------------------------
%	EDUCATION SECTION
%----------------------------------------------------------------------------------------

\section{Summary}

I am a software engineer with over 12 years of professional experience. I have worked on desktop, mobile and embedded applications. I have strong knowledge about C, C++ and development of graphical user interfaces using Qt/QML. I have a solid background in creating complex mobile applications for Android. I have extensive experience in the full life cycle of software development right from requirements gathering, design, prototype, managing schedule, development, testing and maintenance.

\section{Areas of Expertise}

\cvitem{}{C++ programming skill and experience}
\cvitem{}{Mobile applications development for Android and Tizen}
\cvitem{}{Expert in working with various operating systems}
\cvitem{}{Skilled in the development of cross-platform and embedded applications}
\cvitem{}{Experience with Qt Framework and programming graphical user interfaces with QML}
\cvitem{}{Familiar with Agile development techniques}

\section{Technical skills}

\cvitem{OS}{Linux/Unix, Windows}
\cvitem{programming}{C, C++/Qt, Python, JavaScript, Java}
\cvitem{databases}{SQLite, MySQL, Postgres}
\cvitem{version control}{GIT, CVS, SVN}
\cvitem{process}{SCRUM}

\section{Education}

\cventry{2005 -- 2008}{Bachelor of Information Systems}{UNISUL}{Santa Catarina}{Brazil}{}

%----------------------------------------------------------------------------------------
%	WORK EXPERIENCE SECTION
%----------------------------------------------------------------------------------------

\section{Experience}

\cventry{Sep/2018 (current)}{Sr Engineer}{\textsc{Scania}}{Sodertalje - Sweden}{}{
\begin{itemize}
\item Developing the HMI system using C++/Qt.
\item Acting as a PO for the new digital platform for trucks.
\end{itemize}}

\cventry{mar/2016 (aug/2018)}{Sr Software Developer}{\textsc{Symbio}}{Stockholm - Sweden}{}{
\begin{itemize}
\item Developed a security social network for Android.
\item Created a prototype of smart security system using RaspberryPi and C++/Qt.
\item Developed an audio streaming platform including back-end, Android and iOS apps.
\item Implemented an android application for a location based news platform.
\end{itemize}}

\cventry{nov/2014 -- mar/2016}{Software Developer}{\textsc{Nokia Institute of Technology - INdT}}{Manaus}{}{
\begin{itemize}
\item Developed a cross platform application(Qt/C++) interacting with data storage device.
\end{itemize}}

\cventry{aug/2013 -- nov/2014}{Software Developer/Team Leader/Project Coordinator}{\textsc{Samsung Eletronics - SIDIA}}{Manaus}{}{
\begin{itemize}
\item Porting Android application to Tizen platform.
\item Developed Samsung's driving solution that allows users to interact with the mobile device(Android) using voice control features. Worked on this project as Technical Leader and Team Leader.
\item Developed Android mobile application to manage sales and issue an electronic receipt used by Department of Finance of Brazil.
\end{itemize}}

\cventry{jun/2012 -- aug/2013}{Software Developer}{\textsc{Nokia Institute of Technology - INdT}}{Manaus}{}{
\begin{itemize}
\item Implemented a cross platform game with cocos2d-x(C++) running in Linux, Android and Windows Phone.
\item Developed a mobile application in C\# to monitor bus location in the city of Manaus.
\item Implemented HotSpot 2.0 support for Nokia mobile platforms.
\item Developed the embedded software for a portable GPS device to monitor wildlife. The developed SW included advanced features for reduced battery consumption. This platform is now in trial by WWF to monitor animals in Nepal.
\item Two proposals for projects to the internal innovation process promoted by INdT were approved to be initiated:
\begin{itemize}
\item Created a mobile solution to help the hearing impaired detect potential danger in the surrounding environment. This application analyses sounds from the surrounding environment and provides visual and tactile feedback to
the user, with an intensity  proportional to the
volume of the captured sounds and a guess of the audio source (e.g., an alarm, a horn) as well as the direction of the audio source.
\item Researched, designed and implemented an open source public transport platform using Python, Nokia Here API and JavaScript with Jquery and Windows Phone Application written in the C\#. This platform will be used to provide traffic related information to Nokia Here.
\end{itemize}
\end{itemize}}

\cventry{aug/2009 -- may/2012}{Embedded Software Developer}{\textsc{Datacom Telematica}}{Florianopolis}{}{
\begin{itemize}
\item Developed cross platform embedded Linux applications for telecommunication equipment written in C/C++.
\item Implemented a front-end(JavaScript + jquery) and embedded server(C + Postgres) to configuring and managing telecommunications and network equipments from Datacom.
\end{itemize}}

\cventry{jan2008 -- aug/2009}{Embedded Software Developer}{\textsc{Digitro Tecnologia}}{Florianopolis}{}{
\begin{itemize}
\item Developed cross platform embedded Linux applications for telecommunications equipments written in the C/C++.
\item Implemented a distributed backup system for audio records. This system is used by Brazilian Federal Police and Call Centers around the world.
\item Developed a call recorder for Call Centers and Brazilian Federal Police.
\item Developed a desktop screen recorder for Call Centers. This recorder is used by Call Centers in Brazil.
\end{itemize}}

\cventry{oct/2007 -- aug/2008}{Mobile Applications Software Developer}{\textsc{Fazion Sistemas}}{Florianopolis}{}{
\begin{itemize}
\item Implemented a trouble ticket system for mobile devices (J2ME). This system was created and used to Global Village Telecom(Brazilian telecomunications company) partner.
\item Created Web Service to communicate with mobile devices in PHP with JavaScript.
\end{itemize}}

\cventry{dec/2006 -- oct/2007}{Mobile Applications Software Developer}{\textsc{WapCash Project - Unisul/Suntech/Techbay/Nexxera}}{Florianopolis}{}{
\begin{itemize}
\item Developed a mobile payment software in J2ME for cellphones. This system was used by
the platform from the project sponsors.
\end{itemize}}

%----------------------------------------------------------------------------------------
%	LANGUAGES SECTION
%----------------------------------------------------------------------------------------

\section{Languages}

\cvitemwithcomment{Portuguese}{Native}{}
\cvitemwithcomment{English}{Advanced}{}
\cvitemwithcomment{French}{Basic}{}
\cvitemwithcomment{Spanish}{Basic}{}

%----------------------------------------------------------------------------------------
%	INTERESTS SECTION
%----------------------------------------------------------------------------------------

\section{Publications}

\cvitem{}{\textbf{ROLLIN, Fellipe}, Ueno, Alexandre, Bonfatti, Talitha Roberta, Velho, Luana In: Capacidade Empreendedora. Florianopolis : Pandion, 2011, v.1, p. 1-200.}

\cvitem{}{\textbf{Fellipe Rollin}, ``Arduino: from prototype to final product,'' in
Free and Open source Software Developers' European Meeting (FOSDEM), (Brussels), feb 2013.}

\cvitem{}{\textbf{Fellipe Rollin}, ``Arduino: da prototipagem ao produto final,'' Conferencia Latino-Americana de Software Livre (Latinoware), (Foz do Iguacu), oct 2012.}

\cvitem{}{\textbf{Fellipe Rollin}, ``Arduino: da prototipagem ao produto final,'' The Developer's Conference (TDC), (Goiania), Aug 2012.}

\cvitem{}{\textbf{Fellipe Rollin}, ``Arduino: da prototipagem ao produto final,'' The Developer's Conference (TDC), (Florianopolis), oct 2012.}

%----------------------------------------------------------------------------------------

\end{document}